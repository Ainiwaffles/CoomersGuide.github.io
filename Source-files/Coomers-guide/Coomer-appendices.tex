\documentclass[Coomer-main.tex]{subfiles}

\begin{document}

\appendix
\addpart{Appendices}
\label{part:app}
\chapter{Species Descriptions}
\label{appendix:species}

The World Info section of v3 was constructed with a very specific approach in mind, that being a higher-level overview of the methodology behind constructing effective and high-quality World Info entries.
While this is good for the purposes of the guide in terms of adherence to its overall tone and objectives, it also has a major deficiency.
That being concrete examples and deeper guidance on exactly what the application of the discussed strategies looks like is entirely missing, except for an example provided for one use case in the Tips and Tricks section.
This supplement is designed to remedy this problem by providing an in-depth and example-heavy rundown of how to compose World Info entries according to the tenants put forth by the guide.

Since this is going to involve examples taken from my own story, I do have to mention it’s pretty much all monstergirl stuff.
That’s just what I do, so it’s where all my most up-to-date world info entries are.
That said, there’s no reason you can’t take the examples I’ll use here and apply them to whatever it is you want to write.

Starting with species descriptions, since they’re a bit simpler, and because you’ll need to make a species entry for a monstergirl before you can give her a character-specific world info entry.
Like I said in the tips and tricks section, the keys here are very simple, consisting of the species name and its plural.
(In the case of Oni or Arachne they’re innately pluralized, like deer, so you don’t need a second key for plurals.)

\section{Mothgirls}
\label{appendix:mothgirls}

We’ll start with my World Info species entry for Mothgirls, a timeless classic.

\begin{WI}{Mothgirl, Mothgirls}
Mothgirls are monstergirls with soft fur covering their bodies, fern-shaped antennae on their heads, and large wings with peculiar patterns.
Mothgirls have a gentle disposition and high sex drives.
Mothgirls can release shimmering powder from their wings when aroused, which has an aphrodisiac effect.
\end{WI}

This is a good example of what I meant when I said “World Info entries should always be composed as singular paragraphs with short sentences that each individually utilize one of the keys for the entry” in v3.
You’ll also notice it’s very short, which is what I meant when I said entries should be “Sharp and descriptive, while remaining concise”.
There’s no real need to detail anything beyond this very basic level of to-the-point prose, since everything that would involve more specifics is going to be handled through other functions.
A species entry lays down a few concrete, unchanging facts and then ends.
Less is more.

\section{Scylla}
\label{appendix:scylla}

For another example, consider my entry for Scylla (Side note: I have no idea if Scylla is inherently pluralized or not.
I’ve tried to look for answers, but I can’t seem to find any definitive conclusions.
The usage of “Scyllas” seems almost nonexistent though so I’ve chosen to treat it as inherently pluralized.)

\begin{WI}{Scylla}
Scylla are monstergirls with the upper half of a human woman and a lower half made of eight large octopus tentacles.
Scylla tentacles are quite dexterous, constantly slick with wetness, and very sensitive to touch and pressure.
Scylla love to wrap their tentacles and arms around men they fancy, especially during intercourse.
\end{WI}

You can see the conventions of my World Info construction a bit more clearly now.
Always repeating the key phrase in the sentence, keeping language curt but informative, keeping entry size to a minimum, that sort of thing.
Combined with my Tips and Tricks notes on species entries, this should give you a good idea of how to pull these off for pretty much any kind of monstergirl/non-human.

\chapter{Recurring characters}
\label{appendix:characters}

Now, onto the trickier subject, and the much more requested one: World Info entries for recurring characters.
As a general rule of thumb, I never make character World Info entries until I’ve had at least one encounter with them, and I know I want to bring them back up again.
This is just personal preference however, and you’re more than welcome to start a character entry the second an encounter starts, and you have the information to do so.

Sticking in line with the species entries mentioned above, here are the characters that reference those.

\section{Khima}
\label{appendix:Khima}

I’ll start with the Oni of my story, since I mentioned my species entry for Oni in v3 and people seemed to like that entry.
For all of these, the only key for the entry is their name.
(But if a character has nicknames or pet names, they need be made keys as well.)

\begin{WI}{Khima}
Khima is a gigantic Oni.
She stands almost eight feet tall, with plenty of muscle tone across her entire body.
Khima’s skin is a dark tan, contrasted by smooth waist-length white hair.
Khima's eyes are a piercing amber.
She loves drinking copious amounts of alcohol as a way to test the might of men she fancies, but she also has a slightly tender side.
\end{WI}

As you can see, species entries and character entries are constructed with similar methodologies, but I let character entries stray a little bit by comparison.
Still though, you want your entries for recurring characters to focus on broad and concrete things like their species, their height, a very light rundown of their features, and one or two pieces about their personality.

\section{Candy}
\label{appendix:candy}

Our next example is one of my most commonly recurring characters, Candy the Mothgirl.
I’ve had three fully detailed encounters with her, so her entry is almost down to science in terms of potency.

\begin{WI}{Candy}
Candy is a Mothgirl.
Candy has tan fur and soft velvet wings with black and purple patterns on them.
Candy has long and dark purple hair that frames a kind and gentle face.
Candy loves wearing robes that accentuate the impressive curves of her soft body.
Candy is a mature mother figure that treats all things in her life with enthusiasm and kindness.
When in a state of arousal, Candy’s maternal charms mix with her deep lust and assertive nature to make her a powerful force in the bedroom.
\end{WI}

Again, we break a few conventions in terms of the lengthier sentences there, but trust me when I say this shit works very well.
The AI already has some sort of internal dynamite trigger when it comes to handling MILFs, but this World Info entry paired with similar language in the /remember hands me solid gold every single time.

\section{Cherry}
\label{appendix:cherry}

Finally, we have the character entry for my recurring Scylla character, Cherry.
Cherry’s entry is weird compared to the rest, since I mostly use it for details surrounding our specific arrangement and her unique erotic mechanic of love juice.

\begin{WI}{Cherry}
Cherry is a scylla with short brown hair and impressive curves.
Cherry is bonded mates with Ryan, which makes her one of his regulars so she can vent her lust at frequent intervals.
Because Cherry is a scylla, her waist is where her human half transitions into eight blue tentacles of impressive size.
Cherry's tentacles and mouth produce love juice.
Love juice is filled with hormones and nutrients that increase arousal and stamina when ingested.
\end{WI}

You’ll notice several breaks of convention here.
Physical details are largely omitted, erotic mechanics are discussed inside a character entry, and details from a species entry are re-stated.
This is because, while the AI is incredibly good at handling tentacles, it struggles quite deeply with remembering that they’re supposed to be attached to the lower half of a woman and not trying to molest her as well.
Reinforcing these facts across two entries and a lot of /remember helps to alleviate this problem.
Love juice is also something specific to Cherry, and the functionality is too simple to merit an entire separate entry, which is why it’s included here instead.

Sometimes your World Info entries will need to break convention based on the unique demands or challenges of working with a specific character or species, and that’s okay.
Adaptation is at the heart of good writing with AI Dungeon.
Just keep in mind that what I said about the AI’s struggle with physical detail still applies heavily to these entries.
If the details you put here don’t show up, or show up incorrectly, don’t immediately assume it’s because you’re doing something wrong.
The AI is just like that sometimes, tossing any and all mention of deeper detail in the trash to instead yammer about “emerald eyes” for the billionth time.
On the flip-side, it can also pull details exclusively mentioned in your entries out of the depths and make them the focus of a scene, which is incredibly cool when it does happen.

So don’t dismay if it takes a while for you to get a sense of what good World Info entries look like.
I’m tweaking and improving mine all the time, even as I composed this supplement.
People tend to pain pretty deeply over World Info entries, but try not to break any bones over it.
World Info helps, but unlike a lot of other functions, the penalty for misuse isn’t that steep.


\end{document}
