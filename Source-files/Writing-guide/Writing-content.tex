\documentclass[Avsfag-main.tex]{subfiles}

\begin{document}

\epig[Anon][]{I just wanted to play cops and robbers, game. Are you really gonna make me take my dick out again?}
\chapter{Introduction}

Welcome to the /AIDG/ writing guide, written by Avsfag, with contributions from
\href{https://www.google.com/search?q=Severely+autistic+children\&source=lnms\&tbm=isch\&sa=X\&ved=2ahUKEwjMyu-tpYXrAhWSGs0KHStBDwYQ_AUoAnoECA8QBA\&biw=1792\&bih=948\&dpr=2}{special viewers like you}.
This guide is intended to be a living document, and updated as new information/updates come out, but more importantly: \emph{it is not gospel.}
It's possible information in this guide is outdated or downright wrong -\/- that's the risk you run when you attempt to define the behavior of an unfinished application run by a highly-advanced AI.
More importantly: this guide is about \textbf{writing for results}.

That said, I hope this guide is useful to you regardless of whether you're off fighting werewolves or fucking them.
I've had incredibly good results in my sessions with the AI, but make no mistake: it's not like this guide will turn the Mormon's machine into a full-fledged author.
The AI will frequently get lost mid-sentence; it will forget characters; it will throw aliens at you when you're expecting horses.
It will always be a semi-retarded babboon, and if you can't tolerate that, this guide is not for you.

\epig[Anon][]{The wolves fear the coomer mutant.}
\chapter{Setting Up}

Strap in, slip those pants off, and get comfy, because we're going to get started.

First, pull up the game itself by visiting
\href{https://play.aidungeon.io/}{https://play.aidungeon.io/} and making an account.
An account will let you save your adventures, which will be important in cultivating a rich, immersive experience.
Second, dispel the idea that if you follow this guide it'll be a seamless process with perfect results.
It will not be, and you will have to work to make your adventure fun and coherent. But if you write well and work with the AI, you will be heavily rewarded.
Make no mistake, the AI can either be extremely good, or extremely bad.
In order to steer the AI towards being extremely good, there's some settings we first have to tweak before you get writing.

\section{Randomness}
\label{sec:random}

\menu[,]{Hamburger menu left side, Settings, Game Tab, Randomness}\smallskip

The randomness setting controls how random the AI will be when responding to user input.
The higher this value, the higher degree of randomness.
I have mine set between \textbf{0.8 }and \textbf{1.2}.
This is because the AI is already fairly random, and the less random, the better you'll be able to steer it in your adventure.
But you still want a bit of variety.

\section{Length}

\menu[,]{Hamburger menu left side, Settings, Game Tab, Length}\smallskip

The length setting controls how much text the AI puts out per response.
This value is its \emph{maximum} word count (meaning it could also put out less than your maximum value).
The higher this value, the more likely the AI is to lose track of things, and, depending on your randomness value, the higher the chance of a completely random/incoherent response.
I have mine currently set to \textbf{69}.
Tweak these values as needed, and keep in mind, the AI has a limited memory.
This is especially important if you're on Griffin.

\section{Setup TL;DR}

Set randomness between 0.8 - 1.3, and \ul{set length to whatever you please, keeping in mind the longer this value the higher the chance of random output}.

\epig[Anon][]{I CANT STOP FUCKING SALAMANDERS}
\chapter{World Info}
\label{ch:wi}

\menu[,]{Start a game, Hamburger menu right side, Edit Adventure, Scroll down, World info}\smallskip

\wi is a new mechanic in AID that allows you to create static (but player-editable) world information.
For our purposes, we're not going to use it just for world info, but for character information as well.

\section{How is World Info Used for Writing Well}

The AI listens for keys (think tags) in your input or its output.
These keys have entries associated with them that will help guide the AI in making its response.
This is essentially where you define character information like description, allegiance, personality, attitudes, behavior, age, and whatever you want to put here.
You can also put setting info, plot details, etc, but we're focusing on characters for now.

What I do first in a new game is define important characters with no less than three keys that are relevant to the character.
In the ``Entry'' box you have limited text space (about 500 characters) to describe your character or setting, so try to match efficiency with beauty in defining everything you want the AI to know about your character.
Because the AI is retarded, it's best to be direct in describing the character, working in your keys into the entry box.
Here's an example of what a good character might look like:

\begin{WI}{Anon, You, Boy}
You are a 20-something-year-old human male, hailing from Larion.
You're tall and corded with muscle from years of hard manual labor on his family's farm.
You've got long brown hair and emerald-green eyes.
Your parents passed away at a young age, leaving you in the care of your cruel aunt.
You carry a silver dagger and an intense disdain for the undead.
You hope to join a hunting guild one day.
You call your faithful bloodhound \hl{Landy}, which you are very fond of.
\end{WI}

In the above example, notice:

\begin{itemize}
\item
  I am referring to Anon as ``You'' at the start of sentences to really drive it home to the AI that \emph{these are Anon's attributes.}
  Bad writing IRL?
  Yes.
  In AID?
  No
\item
  I am trying to be brief, yet descriptive (characters are precious)
\item
  Defining base character traits, or traits I want the AI to know about ``You''
\end{itemize}

It'd also be important now to add a \wi entry for Anon's (Your) dog, Landy, so the AI can associate Anon and Landy together and make cross-references.
Make sure you talk about two character's relationship if you reference another character in the \wi entry box (see above).

Having a \wi entry for yourself and important characters/settings is important in writing well, because it helps the AI put out more consistent responses for you to work with.

\textbf{Adding an entry for yourself is highly recommended, and continuously adding new entries for characters, locations, plot details etc., makes for a more consistent experience.}

\section{World Info TL;DR}

\begin{itemize}
\item
  Use no less than 3 keys per world info entry
\item
  Write descriptively, yet efficiently, in the entry box. You have 500
  characters to describe a character's personality, characteristics,
  description, and relationships to other characters
\item
  Add a world entry for yourself using ``you'' as one of the keys, then
  refer to yourself as you in that entry
\item
  Have inter-linking world entries that reference each other
\item
  Despite all of this, the AI will sometimes ignore your world info -\/-
  that's where the real work of the writer begins
\end{itemize}

\epig[Anon][]{That requires self-restraint. As a coomer, I don't have that}
\chapter{Remember Pin}
\label{ch:rm}

\menu[,]{Start a game, Above the input box, The thumbtack button (or type /remember)}\smallskip

The remember pin is a powerful tool because it's permanent stuff the AI remembers, but the caveat is you only get about 1,000 characters.
I treat the remember pin as a relatively dynamic place to store currently relevant information (meaning it's going to change a lot).

For result-focused writing like we're aiming for, my remember pin has two or more things in it at minimum:

\begin{enumerate}
\item
  Currently relevant information about characters that references that character's \wi entry if possible.
\item
  Information about the current scene, with rich, albeit brief descriptions of the scene and which characters are in the scene, and what they're doing.
\end{enumerate}

See what I'm doing here?
See how all these tools and static/dynamic data points reference each other?
The remember pin helps you build a web of understanding for the AI, so it has as much rich prose/information to use when generating a response for you.

\section{How to Write for the Remember Pin}

Remember pin gives you a limited number of characters, so try to be descriptive without clutter, and for fuck sake, seperate your entries into paragraphs, and be objective.
Refrain from using ``you'' or ``your'' in the remember pin.
Build the AI a self-referencing web of information for it to generate content out of.

\subsection{Writing Currently Relevant Information}

Think about the scene.
What parameters about the scene's \wi entries does the AI need to know to come up with a sane response?
If your love interest Becky is currently on the swim team (and make sure that ends up in her \wi entry) and is about to start a race, you want to describe her \emph{current} appearance, feelings, behavior etc.
Her overall general appearance goes in world info, but her current look goes in the remember pin.
For example:

\begin{/rm}
\hl{Becky}, \hl{Anon's} 18-year-old girlfriend, is stripped down to her blue one piece swimsuit.
She paces nervously at the edge of the pool, its lapping surface reflected in her goggles.
\hl{Becky} wears a blue swimming cap that holds back her auburn hair.
Her heart speeds with excitement at the thought of racing against the girls from RichVille High.
\tcblower
\em \small If you were a coomer it'd be appropriate to describe how her toned, pale ass swallows up her one piece swimsuit, how her sculpted breasts bulge against the rubbery-material, leaving little to the imagination, etc.
\end{/rm}

This follows the same pattern from writing \wi in that you practically always begin the sentence with the subject, and then next sentence the subject's pronoun.
The AI can get confused easily — in actual writing this is bad form, but when writing with the AI, you need to hold its hand like this.

\subsection{Writing Current Scene Information}
\label{subsec:currect info}

In a new paragraph, store \emph{current} scene information.
Expect to be changing this entry a lot as your story progresses, otherwise it'll confuse the AI.
This is kinda similar to relevant information above, only this is less focused on the character's details and more focused on the current scene's information.
Again, using the same style of drilling the subject into the AI's head, you should craft an \emph{objective summary} of the scene.
Using the swim team example from above, here's how that might look:

\begin{/rm}
\hl{Anon} is at PoorVille High School to cheer on his 18-year-old girlfriend \hl{Becky}.
Dressed exclusively in blue PoorVille gear, with a giant foam finger to boot, \hl{Anon} watches \hl{Becky} nervously pace around the pool.
Girls from RichVille High strut arrogantly from the locker room, dressed down in their own golden one piece suits.
He's doing his best to give \hl{Becky} his full support.
\hl{Anon} wants \hl{Becky} to win the race.
\tcblower
\em \small Again, if you were a coomer, maybe say, shit like ``Anon can't wait to bring Becky back behind the school and fuck her tight pussy in her one piece.
His eyes flit toward Becky's slender legs, but are also drawn to the chests of the girls from RichVille.''
\end{/rm}

\section{Remember Pin TL;DR}
\label{sec:rm tl;dr}

\begin{itemize}
\item
  Remember pin is for objective, largely dynamic (changing) information, with a larger character limit than \wi
\item
  At minimum, remember pin should have two things in it:

  \begin{itemize}
  \item
    \begin{quote}
    Currently relevant information about the character(s) --- their descriptions, feelings, current actions and behaviors, etc
    \end{quote}
  \item
    \begin{quote}
    Current information about the scene --- what's happening, who is involved, what the possible outcomes, and what the character's objectives are
    \end{quote}
  \end{itemize}
\item
  Throughout all of this, try to reference corroborating information in \wi
\item
  Remember pin is information that will change often, so don't forget to either clear it or modify it
\end{itemize}

\epig[Anon][]{Respecting women is addictive}
\chapter{Important Technical Details}
\label{ch:techinfo}

\extrafootertext{Important note: \aid is always updating --- this information is liable to change often.}

\begin{itemize}
\item
  The AI responds using GPT-3 models. The first response from the AI after the prompt is in GPT-2, but everything after uses GPT-3.
\item
  According to the FAQ section,
  \href{https://aidungeon.io/frequently-asked-questions/}{\emph{The AI
  can only remember back
  }}\href{https://aidungeon.io/frequently-asked-questions/}{\emph{\textbf{10
  action-result pairs}}}. In order to keep the AI from forgetting things
  use \nameref{ch:wi}, the \nameref{ch:rm}, and \hyperref[ch:writing]{general writing advice}.
\item
  Remember pin is limited to 1,000 characters and \wi entries are limited to 500 characters
\end{itemize}

\epig[Anon][]{Got antsy to finish my coom sesh with a longer scenario and it turned to a quad murder/rape.}
\chapter{How to Write For Results}
\label{ch:writing}

Finally, I got to the fucking point, and at the fucking point is where I expect to lose 70\% of you.
Let's get this out of the way first: if you want to get great responses from the AI, you have to treat the whole thing as a writing game/exercise.
It is not a ``mash enter and get gold'' situation (though sometimes hitting enter is okay), and \emph{the AI will only get you halfway there}.
Expect to spend a lot of time re-rolling responses, editing the AI's output, and writing in a specific style to create coherent, fun adventures.

Here's a few things about the AI when it comes to user's writing:

\begin{description}

\item[The AI rewards rich, detail-oriented writing]
If you can write well already, you're in luck.
The AI likes it when you get really specific about things, from the look on a character's face to the style and brand of shoes they're wearing.
That said, you're still going to have to write in a specific style (covered below).

\item[The AI likes to read things it already knows]
By this I mean the AI likes it when you reference things in \wi, the remember pin, and in the past 20 or so lines.
Really drive home details and characters.

\item[The AI has a short attention span]
Even if you're writing with a focus and following the tips in this guide, the AI might forget your characters entirely, or mix up their pronouns/who is speaking.
This is just how the AI behaves.
Luckily, this guide + edit will help you keep the AI focused and on-track.

\end{description}

\section{Starting Out and Guiding the AI}

As you're starting out with a scene, do two things: open up your mind to the possibilities of what might happen, and have a limit for what you're
going to \emph{allow}.
This will give you a guideline for how much control to exert over the AI, as well as room for the AI to throw out some interesting twists and responses.
Let's look at an example summary from an Anon in the thread before we deal with actual output:

\memearrow{gentle slap my gf}

\memearrow{AI makes her cry and scream for someone to come save her, while describing in detail how her skin was turning red and then purple}

First, assess the outcome of the above scene in terms of your constraints for realism.
This might be a totally acceptable response from the AI in your mind, and it might even click with your \wi.
In that case, Godspeed you crazy bastard.
But for many of us, this might be silly and nonsensical.
You have a few options in guiding the AI:

\begin{enumerate}
\item
  Use the pencil icon to edit the AI's response and make it milder, and
  then use \say, \Do, \story to respond to the AI and keep the
  plot/scene moving (recommended).
\item
  Counter the AI with \story. This could be something like:

  \emph{You rub your eyes as if you were waking up from a bad nightmare.
  Stella's skin isn't changing color; it's as pale as it's always been.
  And she's not screaming, but she is crying, and somehow, that hurts you more.}

  Just be prepared for the results.
\item
  Re-roll and hope for a saner response.
\item
  Hit \keys{\return} and just roll with it.
\end{enumerate}

The takeaway is that regardless of how carefully-crafted your \wi, remember pin, and settings are, the AI is still entirely capable of fucking shit up.
Be ready to hold its hand.

\section{General Style}

You putting a leash on the AI is only half of your job. The other half is writing well, and writing in a way the AI can understand.
Writing with rich details and explicitly referencing characters and
locations/plot details involved in the scene (who have world info and
remember pin entries) is the best way to get the AI to put out a
response that falls in the realms of possibility (however you've defined
those in the section above).

It breaks several writing conventions, but I've had GREAT results with
the following formula:


\begin{storyb}
	\say You're deluded, \hl{Dr. Evil}, if you think \hl{me}, \hl{Mommy},
	\href{https://www.youtube.com/watch?v=TDpXdCQIE1Q}{\hl{Goose}},
	and \hl{Mormon} are going to let you escape your \hl{cave of evil} without a fight!
\end{storyb}

\begin{storyb}
	\Do I step to the side, dodging \hl{Doctor Evil}'s disintegration ray, and at the same time I draw my pistol and aim it squarely at \hl{Doctor Evil}'s chest.
\end{storyb}

\begin{storyb}
	\story Laughing maniacally the \hl{evil doctor} fires his deadly disintegration beam right at \hl{you}, \hl{Mommy}, \hl{Goose} and \hl{Mormon}.
	\hl{Mommy} and \hl{Mormon} roll to the side and evade the beam, but \hl{Goose} isn't fast enough — you watch helplessly as \hl{Goose} glows bright red.
	A scream escapes his lips before he explodes into ash.
\end{storyb}

Notice how I've highlighted every character that has a \wi entry, and a scene location that it's in the remember pin?
This gives the AI far more context.
It's also good practice to effectively lead the AI along.
Look at my \story example, and pay attention to what happens to Goose when he's hit with the beam:

\emph{Goose glows bright red. A scream escapes his lips before he explodes into ash.}

This way the AI knows Goose dies, glowing bright red, and can add in its own flavor.
Maybe Goose's ash comes together to form a new Goose?
Maybe Mommy and Mormon have something to say about Goose dying?
Give the AI some room here to keep you on your toes as well.

Also take note of how often I'm referring to the characters in the scene, which is pretty much every sentence.
It's bad writing practice IRL, but good for the AI, as it will literally forget who is in the scene sometimes.
This technique, combined with rich writing and hand-holding, produces a higher quality scene. 
And your imagination/threshold for bullshit is the limit for what can happen in AI Dungeon.

Remember:\textbf{In general, the more detailed and explicit your input, the better the story overall.}
\emph{\textbf{You are telling a story,not playing a game.}}

\section{/Say}

This is what you, your character is saying.
Try to be explicit and natural in your dialogue, as if a real human were saying it, \emph{with} the caveat that you should mention another character's name in the dialogue.
Again, this is bad practice in actual writing, but in AI writing it's a good idea.
Some bad dialogue that leaves the AI open to a ton of interpretation:

\begin{storyb}
	\say What's down there?
\end{storyb}

It's short and extremely open-ended.
It gives the AI too much room for randomness and it doesn't know who you're trying to address. 
Most of all, it's just lazy.
Instead, this was collected from the thread, and generated a great response from the AI:

\begin{storyb}
	\say \hl{Emma}, how's the cargo looking? Everything secure?
\end{storyb}

This is so much better! It's natural sounding as far as objective
critiques of dialogue are concerned, and what's more -\/- it narrows the
AI's responses. The AI then knows its next response should detail the
cargo and its status, and should involve Emma in some way. Emma probably
has a \wi entry, and relevant scene information is present in the
prompt and remember pin.\footnote{\href{https://imgur.com/a/tDOMioc}{Here}'s the AI's full response if you're interested}

I wouldn't edit a single thing out of the AI's response to this Anon's dialogue, nor would I re-roll.
Instead I'd totally roll with it.
It also doesn't hurt that this Anon has been explicit in his writing and in describing the setting and prompt.
Remember, \textbf{the more detail the better.}

\section{/Do}

\Do allows you to perform actions and directly impact the output.
It's your arm \st{or dick}.
It can take simple or robust input, and takes it in first person.
When writing for results, this one is really up in the air --- sometimes it's perfectly valid to
say, ``I smite the dragon.''
BUT, if you want to follow along with the pattern of having lengthy, fleshed-out scenes, you should write something that slows the pace of the AI down a bit.
So instead of just telling the AI that its dragon is dead, try this:

\begin{storyb}
	\Do I raise my silver longsword and aim its gleaming tip at the head of the \hl{red dragon}, while beginning the incantation to summon a lightning bolt from heaven to smite the \hl{dragon}.
	As I summon the lightning bolt, I say, ``Your time has come, foul beast!''
\end{storyb}

This gives the AI enough information to know that:

\begin{itemize}
\item
  You have a gleaming silver longsword (maybe that's in \wi?) and
  you're pointing it at the dragon
\item
  The dragon in this scene is red (and depending on your \wi,
  varies in size and behavior)
\item
  There might be a lightning bolt in the next scene
\item
  That you are speaking dialogue for the AI to deal with
\end{itemize}

Again, this is all part of our design pattern: guide the AI along, edit for clarity, and allow for slight, story-valid variation.

\section{/Story}
\label{sec:story}

Your most powerful tool in your arsenal of actions, \story lets you control not just the scene, but the actions and dialogues of the characters in the scene.
This action is incredibly useful for reacting input, or, more importantly, doing things that you don't think the AI is capable of appropriately describing on its own.

\story input should be in second person present tense, just like everything else (except for \Do).
When a scene is stalling out, or getting off the rails, use \story to correct course or progress it and set up characters so they either have input, or they reappear
(\hyperref[ch:qa]{more on this later}).

Of all the actions, this one allows for the most freedom, so write it as you would write prose, with the caveat being you should beat the AI over the head with what character is where, what they're saying, and what they're doing.
If you want to make sure your scenes are rich and full of dialogue and action, follow my advice from the \Do section and drag things out.
Unless you want an action completed post-haste, try to stretch things out.
For example:

\begin{storyb}
\story Shower water cascades over your scarred up back, stinging your many cuts.
With one hand against the wall, the other at your side, you spend at least 30 minutes in the shower reflecting on today's battle.

``How could Chad have been so reckless?
Why did he charge in, instead of waiting for my backup?'' you say, fighting sobs.

You miss \hl{Chad}.

There's a knock at the bathroom door.

``Anon, are you okay in there?
You've been in the shower for awhile.''

\hl{Rena}'s voice drifts to you through the mist, guiding your troubled mind back to reality.

\keys{\return}
\end{storyb}

There's a lot to unpack here, so let's go over it:

\begin{itemize}
\item
  We'd use the remember pin and \story to establish that Anon (you) is
  in the shower after an intense battle, and he's wounded
\item
  Giving Anon a posture --- even in the shower --- could be great bait
  for the AI to get into the nitty gritty of how you're standing,
  behaving, etc
\item
  Telling the AI you've been in the shower for 30 minutes could color
  Rena (or Anon's) followup dialogue
\item
  The self-dialogue is important because it will impact the way the AI
  handles our battle-scarred Anon

  \begin{itemize}
  \item
    \begin{quote}
    When appropriate, make sure you say ``you say'' as a dialogue tag.
    Feel free to add an action or something here, like ``fighting
    sobs.''
    \end{quote}
  \end{itemize}
\item
  Again, telling the AI you miss Chad lets it at least know Chad isn't
  in this scene, \emph{but it keeps him in memory}. This is helpful in
  case the AI forgets Chad entirely (and if that happens see the
  \hyperref[ch:qa]{Q\&A section} for advice)
\item
  Writing Rena's voice as angelic colors her personality for you, and
  tells the AI she's probably going to be more caring in her behavior.
  Rena should hopefully have a \wi entry as well to help the AI
\item
  Driving home that Anon is shaken up by saying he has a ``troubled
  mind'' will help the scene unfold a lot better
\item
  Hitting \keys{\return} at the end of a good \story hook, instead of using
  \story again, will prompt the AI to respond on its own to everything
  you've given it
\end{itemize}

Again, apply as much slowed down, rich scene detail as possible, as well as being direct with who you're talking to, and who is speaking, while cross-referencing the remember pin and \wi data.
But also keep in mind: you're writing for your own enjoyment as well, so make it good.
Make what you want to happen, happen.

\section{Edit}

\begin{storyb}
``Hmm?'' I ask, eyes closed in ecstasy.

``I'm sorry,'' he says and thrusts the knife into my chest repeatedly as I scream.
\end{storyb}

Edit is the pencil icon above your input box.
It's extremely important you use this feature often, because the AI will very often get things wrong.
No matter how explicit you're being, the AI will flip you the bird and do what it wants (but again, that depends on your tolerance for bullshit and what you had in mind for the scene).
It's important that you clean up the AI's output as much as possible.
This means fixing its grammar, spelling, making sure it's getting names right, that the right person is speaking with the right dialogue, etc.
Fixing dialogue tags is especially important because shit can get off the rails really quick if your once chipper friend gets his dialogue mixed up with the evil wizard's.

I use edit to fix and clean up dialogue/grammar/whatever, but I also use it to \textbf{steer scenes}.
A lot of the time you can have the AI give you a random response that seems entirely valid and logical until you get to the last line.
With edit, you can correct the AI.
This involves playing God a little bit, but sometimes you have to practically write the story yourself, and sometimes the story writes itself with you cleaning up after it.

\section{Re-Roll}

This is the button to the far right of your commands.
It looks like a browser's refresh button.
Basically if you're unhappy with the AI's output, refresh the AI's response for something new.

\section{Redo}

Lets you step back in the AI's response and memory.
Did you fuck up big time and roll with it, only to find your waifu is canonically dead?
Redo and use the edit button, king.

\section{Getting Lewd (in progress)}
\label{sec:lewd}

I'm well aware many of you are playing this game just to get off.
And that's perfectly fine --- the AI has many uses beyond just stabbing you in the chest and goosing you.
So let's take a closer look at how to use this writing guide to make this lewd.

\subsection{Foreplay (Getting the AI to get lewd)}

First you have to realize this thing can do practically anything you
want it to do, and, if you properly use the tips in this writing guide,
\emph{so can you}.

\smallskip
\textbf{!!IMPORTANT:}

\menu[,]{Hamburger Menu, Settings, Toggle ``Safe Mode'' off}
\smallskip

Let's apply some of our writing tips, because no matter if you're slaying or laying a dragon, you need to master these skills:

\begin{enumerate}
\item
  Using \nameref{ch:wi}
\item
  Using \nameref{ch:rm}
\item
  Using verbose and rich writing, repeating details and scenes to keep
  them in memory, and then writing the scenes so you don't end it too
  soon.
\item
  Giving the AI room to play around with
\item
  Heavy editing and guidance of the AI
\end{enumerate}

I'll go over how these things apply to lewd scenarios.
Let's start with \wi.

\subsection{Lewd World Info}

Basically this is the same information from the standard SFW but with a NSFW bend.
So put a little more emphasis into physical descriptions, giving the AI some queues.
Take Stacy for example.
She's hot.
But why is she hot?
Maybe you write her \wi entry like this to describe her sexual features:

\begin{WI}{Stacy, Slut, Bitch}

Stacy, with ass-length blonde hair that falls down like spools of gold, is a gorgeous, spunky college student, and the daughter of professor Grey.
Stacy possesses slim, rounded shoulders and milky white skin.
Stacy's breasts are her most prominent feature, sitting at a nice G-cup, adorned with two pink nipples.
Her thin waist pinches inward and then bows out, further exaggerating her womanly curves.
She's got a nice rounded butt that's well-supported by her thick, juicy thighs.
Stacy loves to be spanked.

\end{WI}

Obviously that's insanely basic, but here's the fun part: \emph{you can put anything there}.
Want to describe how Stacy's pee tastes?
If she has pubes?
Put that in there and gooooo.

\subsection{Remember Pin for Lewd Scenes}

Same deal with the SFW remember pin pretty much, only you should bend your writing towards being raunchy. Describe the scene details in explicit detail if you feel up to it.
Describe the way Stacy's breasts strain against the fabric of her shirt, or how her plunge neckline barely covers anything as she laughs and giggles at all of Anon's jokes.
Just take the \hyperref[ch:rm]{remember pin advice} from earlier, and make it lewder, especially as a sex scene starts to unfold.

\subsection{Writing a Good Sex Scene}

Writing a good sex scene with the AI requires all the skills you learned about above in the general writing guide, and for you to do one very, very important thing: slow the scene down.

See, you can really get the AI to get the hint and just start gobbling your cock with a bit of \story magic, but a good sex scene has tension, buildup, and creativity.
The AI (like you) has a ``hair trigger'' for sex, and likes to describe an action or two, cum, and then \hyperref[ch:qa]{Count Grey} stabs you in the chest.
But that's not hot, nor is it satisfying to you.

Though it may read like ill-informed smut, start with flirting and foreplay with the AI.
Treat it like you would a partner, or how your character would treat said AI in your given setting.
Using a mix of \story and \Do inputs, write carefully that you notice his or her \wi relevant sexual characteristics.
So instead of just hugging Stacy, you:

\begin{storyb}
	\story You press your body up against \hl{Stacy}'s\sethlcolor{red} wrapping your arms around her \hl{athletic frame}.
	You can feel her \hl{heavy breasts squeezed up against your chest}, and you're ashamed to admit that \hl{you're starting to feel a little turned on}\sethlcolor{yellow}.
\end{storyb}

What I've done in the above example is highlight a few key scene details that signal to the AI that you're feeling frisky, or want to do a secks with Stacy.
Then the AI will spit out a response on Stacy's behalf, and you roll from there, maybe using a \Do to pinch Stacy's butt playfully or something and try to move things along.

You get the idea.
\emph{Draw. Out. Everything.}
And that includes the actual fucking too, and this is where you need to get creative and make heavy use of the editing tool, because like I said, the AI likes to cum quick and won't even cuddle aftwards.
How do you deal with this?

First, you need to dig deep into the shallow well of your creativity and find the naughty bits that you can use to describe sensory details.
Put into \story the smell of sweat, the feel of her pussy engulfing your cock, the sounds you and Stacy are making, etc. Then use \Do to start moving things around, slap her ass, pull her hair, etc, but fucking DESCRIBE everything.
So if you use \Do to spank Stacy, write some stuff in \Do about how her ass jiggles when your palm snaps off it, how you hit her so hard that \emph{your} palm actually hurts.
That might look like this:

\begin{storyb}
	\Do Pull my arm all the way back and wind up, and then, right as I feel \hl{Stacy}'s\sethlcolor{red} \hl{tight pussy} start to contract and \hl{convulse around my cock in orgasm}, \hl{spank Stacy ass red}.\sethlcolor{yellow}
\end{storyb}

And guess what champ?
The AI will \emph{heavily} reward you when you do stuff like this, and so will your boner and or vagina (pls be in London).
The AI can go multiple rounds and so can you, but given the context of your story, should you?
Maybe it's dangerous to fuck in the woods with a killer about.
Maybe you and Stacy are nobles of rivaling kingdoms and your love is forbidden.
Think about plot too, because the best smut isn't just mashing bits together --- it's letting sex be the literal climax to the plot.

\subsection{Making Sex Last Longer, and Making it Better}

The AI will sometimes orgasm and try to move things along in like… three actions, which is absolutely no fun.
So again, you have to get creative, and you also have to make use of the edit function a lot.
Switch positions, describe the position to the AI with \story, do something lewd.
Pause the sex to advance the plot with \say (this is a good one).
Do stuff like that.
And if the AI leads off a final sentence with: ``I'm gonna…'' then YANK your dick out of her, describing how it feels with \story, and then, in that same \story input, use dialogue tags to tell your AI waifu to beg for it.

Alongside using your writing skills to draw out scenes, make sure you keep referencing your partner's features to add sensory details (bonus points if your partner's features are also in \wi).
And it doesn't always have to be her breasts or his dick --- it can be their glacier-blue eyes or something like that.

Let me finish this section off by reiterating: you can make ANYTHING happen.
You can fuck ANYTHING, in ANY WAY you want.
Want to fuck a giant spider up its cloaca?
You totally can, with the right kind of writing.
Want to trap a fairy in a glass jar and take her to the local bukake bar?
You sick fuck, that's possible, but disgusting.
The better the setting and scenario, the more visceral the details, the better the scenes you get.

Just know the Mormon does have access to your logs, so don't do anything
illegal.

Happy fapping!\footnote{For more details on maximizing the AI's potential for erotic content, read \href{https://guide.aidg.club/}{``A Coomer's Guide to AI Deungeon''}.}

\section{Tips and Tricks}
\label{sec:tips}

\begin{enumerate}
\item
  It's important to occasionally use \story and set the scene again ---
  both for yourself and for the AI. Think of it like a scene summary. If
  you're in the middle of a boxing match, absolutely use \story to
  progress the scene and then recap a little of what's going on. Remind
  the AI.

\begin{storyb}
	\story You bob and weave, ducking low as \hl{Murphy}'s wild right hook flies just above your hairline.
	You're in the most intense boxing match of your life against \hl{Murphy O'Neil}, the Irish heavyweight champion of the world.
	The crowd explodes with unrest as \hl{Murphy} fails yet another attempt to knock you out in one punch as you step back from his blow.
\end{storyb}
\item
  Add in ``feeler scenes'' where you kind of lean back and really take
  in all of what's happening to you in rich detail. You could do this
  with \Do or \story. Just kind of regurgitate the scene and give the AI
  an opportunity to carry the scene next. Again, if you're boxing
  against Murphy O'Neil, maybe use \Do to say you look around at the
  crowd, you take in Murphy's form and study it, reminding yourself of
  his dangerous right hook. This should be stuff the AI already knows or
  might have forgotten due to memory limitations.
\item
  Break up the flow with \nameref{ch:specialinputs}.
\item
  Periodically remind the AI of plot threads you'd like to see
  continued. The AI has a limited memory and will forget important or
  interesting plot threads within 10 actions.

    \begin{storyb}
	\say I can't believe I was \ul{almost kidnapped} by the \hl{man in black}.
	I hope he doesn't find me here in this treehouse.
    \end{storyb}
\end{enumerate}

\section{How to Write For Results TL;DR}

\begin{itemize}
\item
  GIT GUD AT WRITING (seriously, you need to be a good writer first)
\item
  Draw out each and every scene you can with enough detail to kill a
  horse. Don't just eat the cookie --- \emph{take a hungry bite of it,
  savor the rich sweetness of the chocolate or the saltiness of the
  dough, sigh as your girlfriend places another one in front of you with
  a motherly smile on her face.}
\item
  Beat the AI over the head with who is saying/doing what
\item
  Break writing convention and mention another character's name often in
  dialogue to keep them in the scene/AI's memory

  \begin{storyb}
    \story ``Thank you for the sword, Charles,'' you say, taking the
    sword.
  \end{storyb}
\item
  Edit the fucking shit out of the AI's response and make sure you get
  the dialogue right
\item
  Add in ``feeler scenes'' to recap what's going on for the AI
\item
  Re-roll utter bullshit responses
\item
  Play a character with \say and \Do
\item
  Always try to reference \wi and remember pin data
\item
  \href{https://i.redd.it/wtoc2fwzb0v11.jpg}{Hold the AI's hand},
  but also let it have some breathing room. This guide is about telling
  a story, not mashing \keys{\return}. \emph{Sometimes shit doesn't work out for
  you.}
\end{itemize}


\epig[Anon][]{Why interact with family and friends, when the machine can simulate better ones?}
\chapter{Special Inputs}
\label{ch:specialinputs}

Special inputs are special, just like you :) They're inputs that can make the AI do interesting stuff, like generate lists, summarize things, switch perspectives, and more.
To invoke a special input (as far as I know), you use a \story command, and write something like this:

\begin{storyb}
	\story Anna's POV:
\end{storyb}

To switch to Anna's POV. Leave the colons in there as well. This will
switch you to that character's perspective, \ul{but be warned: you'll
have to switch back with a \story ``Anon's POV:'' again, and even then,
you'll need to basically write the AI into remembering you're Anon.}

Here's a few special inputs that people have found:

\begin{itemize}
\item
  \textless character's\textgreater{} POV:
\item
  \textless character\textgreater's thoughts about
  \textless subject\textgreater{}
\item
  \textless character's\textgreater{} brain root directory:
\item
  2nd Person POV:
\item
  Your POV:
\item
  Credits:
\item
  Epilogue:
\item
  In loving memory of:
\item
  Score:
\item
  Deaths:
\item
  Inventory:
\item
  Items obtained:
\item
  Imagine the smell:
\item
  Summary:
\item
  Achievement unlocked:
\item
  The moral of the story is
\item
  This scene had the following effects on you for the following reasons:
\end{itemize}

You can use special inputs in some really creative ways, besides just switching perspectives or getting funny results.
For example, you can have the AI guide the story with an explicit, detailed list.
To do this, use \story, and then describe something another character is doing that surmises the contents of the list.
Cap off the first sentence with a colon (indicating a list), and then fill out the first entry of the list manually by holding \keys{\shift + \return} to enter in a line break without submitting your text to the AI.
Add another dash to prompt the AI to fill it in using the context you provided it in the first sentence + listitem. An example from the thread:

\begin{storyb}
	\story Your \hl{waifu} walks over to you and says, "What kind of sex do you want to make at me, \hl{Anon}?"
She suggests that you:

— Sneak out behind the school, and make love under the bleachers

— 
\end{storyb}

You'll get similar responses in a list, which gives you the opportunity to respond to the AI with a \say: ``Let's sneak out behind the school, and make love under the bleachers.''
You can also use these lists to have the AI recall important plot details or past actions, bringing them back into memory for the AI.
You could \say:

\begin{storyb}
``You think back to everything that's happened in the past hour:

— You were chased by a man in black, but managed to escape

—
\end{storyb}

And then, obviously, the same scene rules from the above sections apply.

\section{More Verbose Special Input}

This one was recently uncovered.
By using \story and then writing ``Description of x:'' (x being whatever you'd like the AI to describe), you can get some really great outputs which will make your story much verbose, and will remind the AI of a ton of important details.
Just look at this example:

\begin{storyb}
	\hl{Description of Lenexa‘s architecture:} Lenexa is situated on a hilly terrain and it's architecture reflects this, most buildings here are three--floor walkups.
The city has stone buildings from the Iron Age (prior to your time).
The architecture of the city is an old idiom of ziggurats (oblong shaped temples), mounds (the hilly terrain) and a type of earthwork ramparts called ramparts (you've never heard of them before).

The city has numerous old wooden buildings, some of which are only one story high and many of which have thatched roots.
They're old and in dire need of repair.
The city also has a great number of old brick buildings which are rare due to the fact that they were only just starting to be built during your time and only in some of the richer areas of the city.

The city also has a high number of small narrow winding streets.
These were made when the hill was built and were necessity to get around the ziggurats.

\hl{Description of the clothing of passerbys on the streets of Lenexa:} Men wear tunics, robes or leather aprons over their clothing.
Most women wear long dresses or tunics over their clothing.
Some people wear cowls or hoods.
Most of the clothing is in dull colours, though some aprons are brightly coloured and robes and tunics have colorful patterns.
Most clothing is practical in nature, with tunics and robes having pockets and aprons and leather clothing having armour plating.
\end{storyb}

\section{Special Inputs TL;DR}

\begin{itemize}
\item
  Special inputs give you new ways to interact with the AI
\item
  Use \story and a colon to start your special inputs
\item
  Use special inputs and lists to have the AI summarize important plot
  threads to bring them back into memory
\end{itemize}

\epig[Anon][]{yfw you will never see a fairy take a bath in a glass of your cum}
\chapter{Q\&A (in progress)}
\label{ch:qa}

\hspace{\parindent}\textbf{Question:} What do I do if the AI forgot a character?

\textbf{Answer:} If you don't have one already, create a
\nameref{ch:wi} entry for them.
Then you need to start writing them back into the scene. The AI has
LITERALLY forgotten them, because its memory is limited.

\story is a great way to do this, but start giving that forgotten character more than just lines of dialogue.
Remember, the AI has \emph{forgotten} that character.
You need to start weaving in their physical descriptions (coinciding with those in the \wi) into your prose, dialogue, the way they slouch, fuck, fight, etc.
Expect to use the edit button a lot here, because the AI isn't going to remember Chad was loyal to you, and he's liable to murder you.
If you're not okay with that, get used to playing God.

\begin{storyb}
	\story \hl{You} and \hl{Stacy} continue interrogating the demon, standing tall and proud against its menacing corporeal form.
	\hl{You} try to puff out your chest and appear intimidating, just like \hl{Chad}, the blond haired alpha male, would have.
	Though \hl{Chad} is off rescuing a maiden, you can still feel his presence right besides you and \hl{Stacy} as you stand before the hulking demon.
	\hl{Chad}'s voice echoes in your head: ``just be urself brah.''
\end{storyb}\smallskip

\textbf{Question:} What if the response is ALMOST perfect?

\textbf{Answer:} Tweak it to make it perfect, this will only help your
future responses.\smallskip

\textbf{Question:} Hey, how can I get the AI to describe a character it
introduces?

\textbf{Answer:} This can be done a number of ways. I highly recommend
you start out with a \Do command that says ``I study X carefully, eyeing
him/her up and down, pausing on each of their unique features to take
them in for myself.''

This will make the AI generate some basic character description. That's
when you take the reins with Edit and clean up its response, and then
hammer home the character's description with a followup \story
interaction/scene. If you like the character, add them to world info so
the AI can take that into account.\smallskip

\textbf{Question:} Have anons found success getting the AI to track and
progress slow body changes?

\textbf{Answer, sourced from \href{https://arch.b4k.co/vg/thread/302217086/\#302219932}{here}:}
What you're expecting is pretty much beyond the AI's capabilities even
now, but it is at least capable of DOING slow changes.

In my experiences, it's good at describing the changes using comparative
terms (ie. "shorter than yesterday") but getting it to consistently
output exact sizes or something like you seem to want would be much more
difficult to do, although it's plausible that it could if you used the
exact same format on a regular basis and did it constantly enough that
the AI would remember and pick up on it

The main thing you should do, though, is definitely have a WI set up
using a common term as the key - like "days pass" and/or "the next day"
if the changes progress daily - if you're wanting the changes to be
automatic, and then a brief description of what those changes are

What I like to do is have a second key that it chains into for "change,
transform, transforming, etc" with a more detailed description of what I
want all the changes to be, and possibly even an end-goal of when the
transformation will be "complete," but while the AI does understand the
idea of gradual change, it's been pretty difficult to get it to be
consistent in scale so unless you directly specify body parts
gaining/losing x inches per day and really hammer it in, you might end
up with one bout of changes being "something looks different but you
can't tell what" and the next one being hugely drastic, so be ready for
that.

At the end of it all, you'll probably still need to use your imagination
a little bit, but hopefully some of that helps.\smallskip

\textbf{Question:} Who is Count Grey? Dendrin? Kyros? They keep showing
up and derailing my adventure!

\textbf{Answer:} These recurring characters are liable to show up at
least once in your adventure. They're likely due to the fact that the AI
was trained on libraries worth of fanfiction, old texts, and other
sources. Count Grey and his ilk are somewhere in there, watching…
waiting… ready at a moment's notice to fuck your story up
(seriously, if you run into any of these clowns it's highly recommended
you edit them out or do something with them).

\epig[Anon][]{Guys I'm getting raped and abused by a female dragon\\ she's cooking my cock and balls with her fire breath}
\chapter{Additional Resources}
\label{ch:resources}

\begin{itemize}
\item
  \href{https://docs.google.com/document/d/1wSz3xlWlqMLFKrLkrO4dNWNZT1vxtrLmq2yGPJj2f4M/edit?usp=sharing}{Google Docs version of this guide}
\item
  \href{https://guide.aidg.club/}{A Coomer's Guide to \aid}
\item
  \href{https://imgur.com/a/mvjk4al}{Discord PDF}
\item
  \href{https://libgen.is/book/index.php?md5=378DC2ACDDFA931EAD042ABAD32943C4}{Self-Editing
  For Fiction Writers} (Libgen download)
\item
  \href{https://prompts.aidg.club/}{Website of /aidg/ prompts}
\item
  \href{https://github.com/FailedSave/storytelling-guide/blob/master/Guide.md}{Storytelling
  GitHub Guide}
\end{itemize}

\epig[Anon][]{Anyone else have a permanent erection?}
\chapter{Changelog}

\begin{itemize}
\item
  26/11/2020 — Guide ported to LaTeX and converted to HTML. Updated formatting for stories and \wi examples.
\item
  17/08/2020 --- update to \nameref{ch:resources} with a r*ddit guide
\item
  12/08/2020 --- minor changes to the \nameref{sec:story}.
  Added a new special input: ``imagine the smell:''
\item
  11/08/2020 --- big day: \textbf{moved the guide to version 1.0!} Added
  another Q\&A about dealing with recurring characters. Started a new
  section in \nameref{ch:writing}:
  \nameref{sec:lewd}. It's in its first draft stage.
\item
  10/08/2020 - added subsection to \nameref{ch:specialinputs} about getting more verbose outputs using ``description of x:''
\item
  8/08/2020 --- new section added: \nameref{ch:techinfo}.
  Added in a TL;DR for \cref{ch:specialinputs}, as well as a new example. Added a new entry for
  \nameref{sec:tips}

  \begin{itemize}
  \item
    \begin{quote}
    Periodically remind the AI of plot threads you'd like to see
    continued
    \end{quote}
  \end{itemize}
\item
  7/07/2020 --- updated the \nameref{ch:wi}, specifically changing the sample world info's pronouns
  from ``he'' to ``you'' in response to overwhelming feedback from the
  thread. Added \nameref{ch:resources}. Added \nameref{ch:specialinputs}. Also
  added \hyperref[sec:tips]{additional tips and tricks}:

  \begin{itemize}
  \item
    \begin{quote}
    Summarizing a scene for the AI
    \end{quote}
  \item
    \begin{quote}
    Adding feeler scenes
    \end{quote}
  \end{itemize}
\item
	8/06/2020 --- changed \hyperref[sec:random]{randomness} output
  values to \textbf{0.8} - \textbf{1.2} in order to compensate for
  extreme AI variation
\end{itemize}

Important thing for me to add later (maybe a technical details section
at the start? Use this to inform your writing)\\

\href{https://aidungeon.io/frequently-asked-questions/}{\emph{https://aidungeon.io/frequently-asked-questions/}}

\textgreater"The AI can only remember back 10 action-result pairs so
anything not in that window will be forgotten unless you remind the AI
about it."

\href{https://mobile.twitter.com/nickwalton00/status/1289974303757201408}{\emph{https://mobile.twitter.com/nickwalton00/status/1289974303757201408}}
(embed)

\textgreater"we limit the context to 1024 tokens"

\textgreater here's an explanation of what a GPT-3 "token" is:

\href{https://nostalgebraist.tumblr.com/post/189212709059/bpe-blues}{\emph{https://nostalgebraist.tumblr.com/post/189212709059/bpe-blues}}

\href{https://nostalgebraist.tumblr.com/post/620663843893493761/bpe-blues}{\emph{https://nostalgebraist.tumblr.com/post/620663843893493761/bpe-blues}}

\textgreater tl;dr: most words are represented as either a single token
or two, so 1024 tokens should be between 500 and 1000 words.

\textgreater The first response uses GPT-2:

\href{https://twitter.com/nickwalton00/status/1289946861478936577}{\emph{https://twitter.com/nickwalton00/status/1289946861478936577}}
(embed)

To add: info Note to Avsfag.

The "taking in rich details" input rarely gives good results in Griffin
and and even if something nice comes out it's usually no more than
sentence long.

Consider adding "Description of x:" to special inputs for people who
want maximum immersion at minimal costs. Screencap from Griffin story,
all AI.

\end{document}
