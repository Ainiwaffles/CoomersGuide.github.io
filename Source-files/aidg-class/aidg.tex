
% This is file `aidg.tex',
% This file is the source of the documentation of the `aidg' class.
% Contact email: latexanoncoomer@gmail.com
% This file is part of ``aidg'' package.
% This file has been placed in the public domain by its author.
% If your jurisdiction does not recognize Public Domain declarations, then either see LICENSE file or visit <http://unlicense.org/>. 

\documentclass[a4paper,12pt,american,oneside]{aidg} % Loads KOMA-Script scrreprt class internally

\hypersetup{ % This embeds XMP metadata into the PDF.
pdftitle={Documentation for aidg.cls document class},
pdfauthor={Coomer},
pdfversionid={1},
pdfpubtype={manual},
pdflang={en-US},
pdfmetalang={en},
}

\begin{document}

\makeaidgtitle{
Documentation for\\aidg.cls\\document class}{}{}{by Coomer\\\href{mailto:latexanoncoomer@gmail.com}{latexanoncoomer@gmail.com}}{1.0}

\tcolortoc

\addchap{Introduction}

This class was created as a personal effort to typeset \href{https://guide.aidg.club/A-Coomers-guide-to-AI-Dungeon/A\%20Coomer's\%20Guide\%20to\%20AI\%20Dungeon.pdf}{A Coomer's Guide to AI Dungeon}.

\aid is available at this \href{https://play.aidungeon.io}{link}.

This is the first version of this class, more customization options will (hopefully) come in the future.

The documentation is bare-bones for now but will be expaned upon with more contributions.

Currently only LuaLaTeX engine is supported.

Github repo for this class's sources is availble at this \href{https://github.com/CoomersGuide/CoomersGuide.github.io}{link}.

\chapter{Macro definitions}

This is a list of all help macros created to ease the typing and representation of various \aid elements.


First and foremost, title cover command:

\begin{aidgdoc}
\makeaidgtitle{Title}{Image options for graphicx}{Image}{Author}{Version}
\end{aidgdoc}

Leaving both image arguments (\#2 and \#3) empty will output text-only title. 
Take note that the empty declareations have to be specified.
For Version please input number only, the word is defined already.

\begin{aidgdoc}
	\tcolortoc
\end{aidgdoc}

Outputs Table of contents, same as \textbackslash tableofcontents.
Visual style was adapted from tcolorbox package.
To obtain the original one, check that package's sources on CTAN.
For now, ToC inserts new page both before and after, this behavior will be customizable.

\begin{aidgdoc}
	\extrafootertext{text}
\end{aidgdoc}

Allows adding a footnote without any marker.

\begin{aidgdoc}
	\aid prints AI Dungeon
	\wi prints World Info
	\an prints Author's Note
	\ans prints Author's Notes
\end{aidgdoc}

Shortcut macros, to save type on typing.
Simple text output, no styling.
\vspace{1em}

Next come environments (boxes) to typeset various \aid's text fields.

\begin{aidgdoc}
	\begin{WI}{#1}
	World Info entry text.
	\end{WI}
\end{aidgdoc}

\begin{aidgdoc}
	\begin{/an}
	Author's Note text.
	\end{/an}
\end{aidgdoc}

\begin{aidgdoc}
	\begin{/rm}
	Memory text.
	\end{/rm}
\end{aidgdoc}

Next commands use menukeys package to typeset text, for easier identification. An example would be: \story.

\begin{aidgdoc}
	\rem prints /remember
	\alt prints /alter
	\Do prints Do
	\say prints Say
	\story prints Story
\end{aidgdoc}

Drop capitals shortcut.
\begin{aidgdoc}
	\lettr Beginning of a paragraph.
\end{aidgdoc}

Epigraph with Author and source. Source may be left blank.
\begin{aidgdoc}
	\epig[Author][Source]{Text}
\end{aidgdoc}

Epigraph without either Author or Source.
\begin{aidgdoc}
	\epign{Text}
\end{aidgdoc}

Intented for makig greentext. Colors text in green (yes).
\begin{aidgdoc}
	\memearrow{Text}
\end{aidgdoc}

\end{document}
